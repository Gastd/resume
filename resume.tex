%% start of file 'template.tex'.
%% Copyright 2006-2013 Xavier Danaux (xdanaux@gmail.com).
%
% This work may be distributed and/or modified under the
% conditions of the LaTeX Project Public License version 1.3c,
% available at http://www.latex-project.org/lppl/.


\documentclass[a4paper, 10pt]{moderncv}        % possible options include font size ('10pt', '11pt' and '12pt'), paper size ('a4paper', 'letterpaper', 'a5paper', 'legalpaper', 'executivepaper' and 'landscape') and font family ('sans' and 'roman')
\usepackage{textcomp}
% moderncv themes
\moderncvstyle{classic}                             % style options are 'casual' (default), 'classic', 'oldstyle' and 'banking'
\moderncvcolor{blue}                               % color options 'blue' (default), 'orange', 'green', 'red', 'purple', 'grey' and 'black'
%\renewcommand{\familydefault}{\sfdefault}         % to set the default font; use '\sfdefault' for the default sans serif font, '\rmdefault' for the default roman one, or any tex font name
%\nopagenumbers{}                                  % uncomment to suppress automatic page numbering for CVs longer than one page

% character encoding
\usepackage[utf8]{inputenc}                       % if you are not using xelatex ou lualatex, replace by the encoding you are using
%\usepackage{CJKutf8}                              % if you need to use CJK to typeset your resume in Chinese, Japanese or Korean

% adjust the page margins
\usepackage[scale=0.754]{geometry}
% \setlength{\hintscolumnwidth}{2cm}                % if you want to change the width of the column with the dates
% \setlength{\makecvtitlenamewidth}{10cm}           % for the 'classic' style, if you want to force the width allocated to your name and avoid line breaks. be careful though, the length is normally calculated to avoid any overlap with your personal info; use this at your own typographical risks...
    % Profile
\name{Gabriel F P Araujo}{}
\address{Brasilia, Brazil}
\phone[mobile]{55 61 982 308 980}
\email{gabriel.fp.araujo@gmail.com}
% \homepage{gastd.github.io}
\homepage{github.com/Gastd}

    \begin{document}
\makecvtitle
\section{Education}
\cventry
{Undergraduate}
{B.E. in Mechatronics Engineering}
{University of Brasilia}
{}
{\textit{Brasilia, Brazil}}
{}
\section{Experience}
\cventry
{February 2013 -- February 2014}
{Software Developer}
{LIPIS/LEI (Laboratory of Instrumentation and Processing of Images and Signals)}
{University of Brasilia, Brasilia, Brazil}
{}
{\begin{itemize}%
    \item Implementation of an autonomous Antibiotic sensitivity testing.
    \item Algorithm previously designed by LIPIS researchers.
    \item The solution uses OpenCV and C++.
    \end{itemize}}
\cventry
{July 2014 -- June 2015}
{Undergraduate Researcher}
{CIC UnB (Computer Science Department)}
{University of Brasilia, Brasilia, Brazil}
{}
{\begin{itemize}%
    \item Development of an \href{https://github.com/bruno147/driver-ga}{autonomous driver} to the TORCS simulator in order to compete in the Simulated Car Racing Championship, a former GECCO Competition.
    \item 5th place in the SCRC 2015.
    \item Confection of a paper describing the pilot development, DOI:~\href{https://doi.org/10.1109/SBGames.2015.19}{10.1109/SBGames.2015.19}
    \end{itemize}}
\cventry
{September 2016}
{Teacher}
{University of Brasilia}
{University of Brasilia, Brasilia, Brazil}
{}
{\begin{itemize}%
    \item Main teacher at ROSJoy Course.
    \item Knowledge network: Robotics, Python, and ROS.
    \end{itemize}}
\cventry
{January 2017 -- February 2017}
{Teacher Assistant -- Computational Fundamentals of Robotics}
{University of Brasilia}
{University of Brasilia, Brasilia, Brazil}
{}
{\begin{itemize}%
    \item Elaborate challenges and assignments.
    % \item under the Professor's supervision for Computational Fundamentals of Robotics course during UnB Summer School and further 
    \item Documentation of the achieved goals.
    \end{itemize}}
\cventry
{May 30, 2017 -- August 21, 2017}
{Software Developer -- Google Summer of Code 2017 participant with GNSS-SDR}
{University of Brasilia}
{University of Brasilia, Brasilia, Brazil}
{}
{\begin{itemize}%
    \item Expansion of the GNSS-SDR software to GLONASS system.
    \item Implementation of both Acquisition and Tracking blocks of the GLONASS to GNSS-SDR.
    \item Further details:~\url{https://gist.github.com/Gastd/f46a2bd78dcc11984e69eb7cbc49f8a4}
    \end{itemize}}
\cventry
{April 13, 2019 -- June 21, 2019}
{Intern}
{LandSense Soluções Tecnológicas}
{Brasilia, Brazil}
{}
{\begin{itemize}%
    \item Embedded software development.
    \item Design and implementation of a Bluetooth mesh protocol.
    \item Main technology: C/C++.
    \end{itemize}}
\cventry
{August 2013 -- Present}
{Undergraduate Researcher}
{LARA (Automation and Robotics Laboratory)}
{University of Brasilia, Brasilia, Brazil}
{}
{\begin{itemize}%
    \item SDR development for mobile robots localization using multi-constellation GNSS systems.
    \item Creation of a \textquotedbl{}chat-bot\textquotedbl{} system for controlling a mobile robot using speech recognition.
    \item Implementation of an indoor localization system using EKF and ARToolKit tags.
    \item Program ROS drivers for GPS and IMU sensors.
    % \item Also engaged in other projects in robotics, more specifically on perception and navigation.
    \end{itemize}}

\section{Computer skills}
\cvitem{Languages}{C/C++, Python}
\cvitem{Frameworks}{\href{https://www.ros.org/}{Robot Operating System (ROS)}, GoogleTest, CMake}
\cvitem{Debugging}{GDB, Valgrind}
\cvitem{Applications}{MatLab/Octave, \LaTeX{}, MS Office, Eagle}

% \section{Computer skills}
% \cvdoubleitem{Languages}{C/C++, Python}{category 4}{XXX, YYY, ZZZ}
% \cvdoubleitem{Frameworks}{Robot Operating System (ROS), GoogleTest, CMake}{category 5}{XXX, YYY, ZZZ}
% \cvdoubleitem{Debugging}{GDB, Valgrind}{category 6}{XXX, YYY, ZZZ}
% \cvitem{O.S.}{Linux (Ubuntu), Windows}
% \cvitem{Fab skills}{Soldering, PCB printing}
% \cvitem{Lab skills}{Soldering, PCB printing}
% \cvitem{Applications}{MatLab, \LaTeX{}, Lyx, LibreOffice, SolidWorks, MS Office, Eagle}

% \section{Activities and Societies}
% \cvitem{UnBall Robot Soccer Team}{Humanoids Group}
% \cvitem{Study Group}{Probabilistic Robotics Study Group}
% \cvitem{Automation and Robotics Laboratory}{AMORA -- Autonomous Mobile Robots Algorithms}

\ 
\end{document}
